\section{Conclusion}

\emph{
I'm fucked lol. \\ \\ 
The Conclusion is the last section of your report (other than Appendixes). In this section, you can revisit the research questions and summarise your answers. Clearly explain how your investigation and your answers are a contribution---why your work is worthy of a passing mark. Also in the Conclusion section, it is good to have subsections that highlight (a) Future Work, in case you were going to keep working on the same line of research or you wanted to recommend follow-up investigation for another student to pursue next year; and (b) Lessons Learned, where you can explain how you might do things differently if you started over, because you've learned valuable things along the way (these could be technical, but they could also be personal, such as organising your time better or listening to the project coordinator who told you to BACK UP your work frequently).\\ \\
FURTHER RESEARCH QUESTIONS THAT CAN BE DONE \\ \\
Research Question 3: Is it possible to retrain the model for other image data sets specifically including images captured from drug screen tests? \\ 
 Research Question 4: Is it possible to retrain the model for other image data sets, specifically including images captured from other cell lines collected from HIPSCI? \\ 
 For these two research questions, providing that the model works for the original datasets mentioned in the 2nd research question, it would be interesting to see if we are able to identify similar rate of change metrics for iPS cells collected from different environments, or iPS cells that were intended for different research purposes. If the model does work for other iPS cells then it would assist in the research that is currently being undertaken in the drug screening community, as well as the King?s HIPSCI community. \\ \\ 
Research Question 5: Is it possible to train the model to perform unsupervised class discovery from the iPS cell images? \\ 
For this research question, this builds on improving the existing model, provided that it works with a range of different iPS cells. It would involve building a feature extraction model on top of, or alongside the existing model that would identify other classes from the population of individual stem cell images. This could even be a clustering problem, I propose that after separating the normal and abnormal cells, to then go back to the normal cells and identify if there are subgroups of cells that exist that can be classified into its own class. \\ \\ 
Research Question 6: Is the model able to work on more complicated stem cell images such as embryo cells or cells collected from PhaseFocus equipment? \\ 
This is also a follow up from the previous research questions, to see if the model works with other types of iPS cells that are may not have the same classes as the ones trained on previously, but would it be possible to implement the feature extraction or clustering problem from the previous question on the new dataset to try and identify a rate of change from one ?normal? or time=0 class and any of its evolutionary stages. This would require retraining the original model with the new classes obtained
}
