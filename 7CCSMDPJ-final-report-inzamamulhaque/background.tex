\section{Background} 


\emph{The Background section of your report should provide the reader with enough technical background so that they understand the area in which your research is conducted. This should be the kind of information that you might find in a textbook that teaches someone about the area. The next section of the report ("Related Work") is where you describe new research in your area, so think of this Background section as where you provide enough information so that the reader will be able to understand the important details contained in the Related Work section. \\ \\ 
For the project, "Using Google's TensorFlow to train the Inception Deep Learning Convolutional Neural Network" in order to classify images of cells from drug and stem cell differentiation screens.?, the data to be used for the research will be images of well plates (using 96 welled plates, of which there are 8 rows and 12 columns) of stem cells that have been collected at the King?s Centre for Stem Cells and Regenerative Medicine. On each image of the plates are stem cells that have been exposed with various levels of Fibronectin (FN) protein of 1 microgram/ml, 5 microgram/ml and 25 microgram/ml. \\ \\ 
Each row of the well plate has a different cell line, and each row contains three wells of the same cell line exposed to the above concentrations of FN. The data is comprised on 50k+ image of 101 cell lines which were imaged once per hour for 24 hours. The lines were imaged multiple times in multiple experiments ? with each cell line being imaged first using a live-imaging device, where the snapshots were taking every 24 hours. At the end of the 24 hours, the same cell line was then stained and imaged using the Operetta High Content Imaging System. \\ \\ 
The Operetta system computes summary plate files, where the cell line from the plate wells and the FN concentration are recorded. The metadata from the saved images then get annotated with the metadata provided from the operetta plate files. They may exist possible issues of batch effects due to errors in plating, and potential variation in the concentration used during the cell assays. The cell line behavior may also differ slightly across multiple experiments, even though they are the exact same cell line. The data was obtained from the following individuals (Domain supervisors) who kindly provided the data privately. \\ \\ 
Dr. Davide Danovi Director, \\ HipSci Cell Phenotyping Centre for Stem Cells and Regenerative Medicine \\ King's College London, \\ 28th floor, Tower Wing, \\ Guy's Hospital, Great Maze Pond, \\ London \\SE1 9RT, UK \\ davide.danovi$@$kcl.ac.uk \\ \\ Dr. Amos Folarin \\ Informatics Software Development Group Leader \\ SLaM/Kings College London\\ amos.folarin$@$kcl.ac.uk \\ \\ Maximilian Kerz, PhD Candidate \\ Lead Developer of RADAR-CNS \\ Front-end Ecosystem \\ Dept. Biostatistics and Health Informatics \\ King?s College London \\ Tel.: +44 (0) 207 848 0924
}
\\ \\ 
Example of glossary entry is \gls{svm}. \gls{cnn}.
Example of bibliography entry is given by Johnstone~\cite{Joh11}.
Further information can be found at: \cite{website:fermentas-lambda}.

