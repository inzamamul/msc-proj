\section{Introduction}

\emph{The Introduction is the first content section of your report. You should describe the general area (e.g., application domain) in which your project research is conducted, the motivation for conducting the research and the overall aims of the research. Be sure to outline your research questions and give a brief summary of the conclusions drawn, though the conclusions will be detailed later in the report. With the Introduction, you want to interest your reader and tell them why they should care about your research and why they should read the rest of the report. The report will be read (marked) by examiners with a technical Computer Science background, but not necessarily any knowledge of your domain, so make sure that you provide enough information for a naive reader.\\ \\ 
My project is titled ?Using Google's TensorFlow to train the Inception Deep Learning Convolutional Neural Network in order to classify images of cells from drug and stem cell differentiation screens?. As part of Google?s efforts to grow its presence in the deep learning space, the company built DistBelief in 2011 as a proprietary machine learning system. The DistBelief system picked up traction among computer scientists wishing to develop and build larger neural networks to answer large data problems. In 2015, Google released TensorFlow, a successor to the DistBelief system, \cite{webiste:https://research.googleblog.com/2015/11/tensorflow-googles-latest-machine_9.html} which was made to be more flexible, portable. \\ \\ 
In 2015, TensorFlow was built upon the DistBelief system and has the added ability to be able to compute any computation that can be expressed by a computation flow graph. The system performs computations on multidimensional arrays, tensors, which are then passed on through neural networks. The system also improves upon DistBelief?s speeds, and scalability. TensorFlow has been made open source from and has been forked now over 25,000 times on the version control repository, GitHub. A Python API is available for developers, scientists and researchers who would like to develop on the project on virtually any domain they see fit. \\ \\ 
Introduction of the Domain and Research Motivation \\ \\ 
The domain is centred around the work being done at the Human Induced Pluripotent Stem Cells Initiative (HipSci) at King?s College London. The group at HipSci are developing assays to image cells in artificial microenvironments to study its responses to various stimuli in order to develop disease models. The HipSci group has state of the art equipment that is able to take pictures of a variety of stem cells for use in training these disease models, and brings together researchers and scientists working in the field of genomics, proteomics and cell biology. This topic is very interesting to me as the developments in machine learning are slowly coming to present itself in this domain to assist in the creation of more powerful and precise disease models. \\ \\ 
Work done in this domain could see the improvements in the way that we diagnose or treat diseases that affect millions of us daily, could see quicker turnaround times that drugs are developed and produced for public health and could help us understand more about how stem cells (specifically induced pluripotent stem cells) behave over time. This is also of great relevance to my domain supervisors, Dr. Amos Folarin and Dr. Davide Danovi, Director of HipSci at King?s College London, and the work done using TensorFlow is of interest to my supervisor, Dr. Nishanth Sastry. Another attraction for this topic is the use of deep learning to provide the analytics and results. \\ \\ 
Deep learning has only recently become a technology trend that more and more domains are turning to, for providing solutions to ever growing complicated questions. I do hope that my experience with this project, especially after using deep learning techniques will allow me to use my skills in other domains as well as the work I will do in the bio-informatics sphere. }
