\section{Approach} 

\emph{The Approach section of your report should describe what you did. You should discuss your research questions in detail here, explaining for each question how you addressed each question (i.e., what techniques you used) and how you evaluated the success (or failure) of your investigation. This should include a description of the data set(s) that you used for your research (e.g., what you included in your Data Acquisition report that was due in March). \\ \\ 
Research Question 1: Is it Possible to Perform Image Segmentation Using the Data Set? \\ The data does not to be cleaned as such, but the images of the cell lines will need to undergo image segmentation in order to extract the individual cell objects so that they can be fed into a neural net pipeline in order to achieve the goals stated in the project description. This can be done a number of ways. Currently one method is to build upon existing open source software that is already built for cell profiling and image analysis (CellProfiler) as demonstrated by Kerz et al., and to include the functionality that allows the user to save individual cell pipelines for feeding into neural networks. Another alternative is to use a U-Net convolutional network for the image segmentation (Ronneberger, et al., 2015) that has seen success in other biomedical image segmentation challenges. The plan is to have the images segmented and ready to fed into a neural net pipeline by the end of April 2017. \\
I used MatLab to segment the images. The process is outlined below. \\
1. The images have been taken every hour for 24 hours using the incucyte imaging device. I will use the cells contained in the first image of the 24 images for training on the model. This will give me the cells that have had the protein added to them. \\
2. Read in the image using MatLab. Identify thresholds of the image. \\
3. Identify a mask of the image to get get a binary image from the gray scale image. This then locates the individual cells, or blobs, in the raw image. \\
4. Save all the cells or blobs into a directory, to be used for training further in the downstream process. \\ \\  
Research Question 2: Is it possible to train a model using TensorFlow to perform anomaly detection on the images, training with normal cells from the iPS image data set? \\ 
For this research question I propose to implement a neural network method for identifying cells that have deviated from the normal trained cells and to then classify as these as abnormalities. The number of cells that will have deviated from normal will increase over time as the cell evolves. It would be useful to keep a scorecard of the ratio of ?normal? to ?abnormal? cells at any time to provide statistics or metrics to identify a rate of change from ?normal? to ?abnormal? that could be useful in learning more about how iPS cells differ from one another. These differences may be due to a variation of genes in the cells or due to some other reason which would be of interest to find out for the purposes of this research. \\ \\
}
