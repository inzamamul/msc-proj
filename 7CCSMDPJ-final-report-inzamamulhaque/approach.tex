\section{Approach} 

\emph{The Approach section of your report should describe what you did. You should discuss your research questions in detail here, explaining for each question how you addressed each question (i.e., what techniques you used) and how you evaluated the success (or failure) of your investigation. This should include a description of the data set(s) that you used for your research (e.g., what you included in your Data Acquisition report that was due in March). \\ \\ 
Research Question 1: Is it Possible to Perform Image Segmentation Using the Data Set? \\ The data does not to be cleaned as such, but the images of the cell lines will need to undergo image segmentation in order to extract the individual cell objects so that they can be fed into a neural net pipeline in order to achieve the goals stated in the project description. This can be done a number of ways. Currently one method is to build upon existing open source software that is already built for cell profiling and image analysis (CellProfiler) as demonstrated by Kerz et al., and to include the functionality that allows the user to save individual cell pipelines for feeding into neural networks. Another alternative is to use a U-Net convolutional network for the image segmentation (Ronneberger, et al., 2015) that has seen success in other biomedical image segmentation challenges. The plan is to have the images segmented and ready to fed into a neural net pipeline by the end of April 2017. \\ \\ 
Research Question 2: Is it possible to train a model using TensorFlow to perform anomaly detection on the images, training with normal cells from the iPS image data set? \\ 
For this research question I propose to implement a neural network method for identifying cells that have deviated from the normal trained cells and to then classify as these as abnormalities. The number of cells that will have deviated from normal will increase over time as the cell evolves. It would be useful to keep a scorecard of the ratio of ?normal? to ?abnormal? cells at any time to provide statistics or metrics to identify a rate of change from ?normal? to ?abnormal? that could be useful in learning more about how iPS cells differ from one another. These differences may be due to a variation of genes in the cells or due to some other reason which would be of interest to find out for the purposes of this research. \\ \\ 
Research Question 3: Is it possible to retrain the model for other image data sets specifically including images captured from drug screen tests? \\ 
 Research Question 4: Is it possible to retrain the model for other image data sets, specifically including images captured from other cell lines collected from HIPSCI? \\ 
 For these two research questions, providing that the model works for the original datasets mentioned in the 2nd research question, it would be interesting to see if we are able to identify similar rate of change metrics for iPS cells collected from different environments, or iPS cells that were intended for different research purposes. If the model does work for other iPS cells then it would assist in the research that is currently being undertaken in the drug screening community, as well as the King?s HIPSCI community. \\ \\ 
Research Question 5: Is it possible to train the model to perform unsupervised class discovery from the iPS cell images? \\ 
For this research question, this builds on improving the existing model, provided that it works with a range of different iPS cells. It would involve building a feature extraction model on top of, or alongside the existing model that would identify other classes from the population of individual stem cell images. This could even be a clustering problem, I propose that after separating the normal and abnormal cells, to then go back to the normal cells and identify if there are subgroups of cells that exist that can be classified into its own class. \\ \\ 
Research Question 6: Is the model able to work on more complicated stem cell images such as embryo cells or cells collected from PhaseFocus equipment? \\ 
This is also a follow up from the previous research questions, to see if the model works with other types of iPS cells that are may not have the same classes as the ones trained on previously, but would it be possible to implement the feature extraction or clustering problem from the previous question on the new dataset to try and identify a rate of change from one ?normal? or time=0 class and any of its evolutionary stages. This would require retraining the original model with the new classes obtained
}
