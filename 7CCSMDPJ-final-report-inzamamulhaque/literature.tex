\section{Related Work}
\emph{MARKING CRITIERA FOR LITERATURE :: }
\begin{itemize}
\item Literature sources primarily come from published, peer-reviewed venues.
\item Multiple literature sources, from different authors/labs, have been critically analysed, compared and contrasted.
\item The relevant literature has been comprehensively covered, both in terms of the Data Science techniques and technologies, as well as the particular domain from which the student's data set(s) and research question(s) originate.
\end{itemize}
\emph{The Related Work section of your report should provide a review of recent literature in the area of your research. This is distinguished from the Background section because it is typically newer and more experimental. If there are standard terms or techniques mentioned in the literature, then you can define what these are in the Background and use the Related Work section to explain how researchers have used the standard techniques as benchmarks or fundamental methodologies for their research. For example, if you review an article that describes using k-means clustering for finding appropriate groups of patients with similar sets of symptoms, then you could describe what k-means clustering is in your Background section and describe how researchers used that technique on patient data in your Related Work section. When you review literature, be sure to explain how the articles you cite are relevant to your project. Be critical---outline pros and cons of the work you are reviewing. Be clear to explain how the work you review is different from your own work. Note that you may find it easier to compare and contrast others' techniques with yours later in the report, after you have explained your own work. That is fine---just be sure to forward reference in the Related Work where you will compare to your own work (and backward reference in the later sections back to the Related Work). This can include information that you had in your Project Proposal report that was due in April, but should typically be substantially expanded from what you had in your proposal. \\ \\
Literature Review An Overview of Data Science Uses in Bio-image Informatics \\ \\
As processing power and efficiency has increased over the recent years, it has become more feasible to conduct large scale analytics on data across various domains. Bioinformatics is one of the domains that has benefited from the statistical methods that can be applied to large sets of bio image data in order to provide both quantitative and qualitative relationships between biological concepts and the data presented. As the usage of deep learning has been increasing in recent times, its applications are starting to appear in bioinformatics. An application of deep learning includes using convolutional neural networks (CNNs) to compare classical features found in bio-images and to then predict the behaviour or actions of cells or small molecules (Chessel, 2017). \\ \\
Utilising deep learning methods in bioinformatics will continue to expand as bio imaging provides for higher resolution images and biological research questions become more detailed. This is an area of research that I anticipate to explore more of in order to develop applications for the above domains. \\ \\
TensorFlow: Biology's Gateway to Deep Learning? \\ \\
 Following on from previous discussions involving the rapid rise of deep learning across various domains, especially in bio informatics. The introduction of TensorFlow to the deep learning community has provided much needed toolset that can better assist in pushing research further. TensorFlow?s ability to provide graphical visualisations as well as quicker speeds for learning (Rampasek and Goldenberg, 2017). Currently the TensorFlow environment is still a low level system, requiring its users and developer to be comfortable working with its Python API. \\ \\
 There are however low level wrappers available that could make it easier for newcomers to begin experimenting with the platform and test it with already acquired computational biological data. Along with its support for parallelization over multiple machines running on either CPUs or GPUs, the bioinformatics community believes that TensorFlow can provide a powerful path to mix deep learning with bioinformatics. \\ \\ 
Machine Learning Predicts the Look of Stem Cells \\ \\
Induced pluripotent stem cells, or iPS cells, are cells that were previously normal cells but have been reprogrammed to behave and appear similar to embryonic stem cells. These cells now have the capability to transform into any type of working cell type used in the body (Scudellari, 2016). Upon its discovery by Shinya Yamanaka at Kyoto University, Japan in 2006, much work has been done in order to try and use these stem cells to cure or treat cell abnormities and diseases found in humans today, among other uses. As these cells provide the ability to be used as a form of regenerative medicine, due to its properties to change into a variety of different cells, it has garnered a lot of interest in the regenerative medical community. \\ \\
Machine learning techniques are starting to make a presence in the modelling and prediction of the behaviour of iPS cells. At the Allen Institute in Seattle, USA a team have been training deep learning algorithms to (Maxmen, 2017) predict what the structure of a cell would look like with minimal data, such as just the location of the nucleus. They were able to do this by identifying relationships between the locations of the cell?s structures of the 6,000 or so iPS cell images that they have in their library. \\ \\ 
A Novel Automated High-Content Analysis Workflow Capturing Cell Population Dynamics from Induced Pluripotent Stem Cell Live Imaging Data \\ \\
One of the main obstacles of performing machine learning techniques on iPS cells is the initial segmentation of the cell from other cells that are found on the cell plates. Using phase-contrast and confocal microscopy to capture bio images of these cells require that many cells be placed on a shared well plate. Images are then taken of the well to capture the various conditions and structures that the cells may take form of. By introducing time as another dimension, i.e. to track the cell structure evolution over time as well as still capturing a high enough resolution image, this may be tricky to complete. \\ \\ 
 A solution proposed is to use a widely used application in the bio informatics community, CellProfiler, to enable identification of iPS cells with highly dynamic structures in single-channel, phase contrast images. The cell segmentation can be challenging at times using conventional methods because of the sometimes poor edge contrasts between the background and foreground, as well as the non-standard structures that the iPS cells can take (Kerz et al., 2016). CellProfiler, built by the Broad Institute at Cambridge, Massachusetts, provides for the ability to conduct the image segmentation and then outputting a pipeline that can be alter used for image analysis applications, such as for input into machine learning methods.
}

Examples of articles we might cite are~\cite{Doe11} and~\cite{JohSil05}.
